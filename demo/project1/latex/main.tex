%! Author = ACA
%! Date = 11-Sep-21

% Preamble
\documentclass[a4paper]{report}
\usepackage[margin=0.75in]{geometry}

% Packages
\usepackage{float}
\usepackage{titlesec}

% Utilities
\input{utils}
%! Author = ACA
%! Date = 11-Sep-21

% Packages
\usepackage{graphicx}
\usepackage{xparse}
\usepackage{xcolor}
\usepackage{fancyhdr}
\usepackage{lastpage}
\usepackage{float}
\usepackage{titlesec}
\usepackage{etoolbox}
\usepackage[style=iso]{datetime2}

\graphicspath{ {./media/} }

% Hyperlinks in tables of content, figures, tables...
\hypersetup
{
    colorlinks,
    citecolor=black,
    filecolor=black,
    linkcolor=darkgray,
    urlcolor=black,
    linktoc=all,
}

% Environment for requirements
\NewDocumentEnvironment{requirement}{m o}
    {
        \vspace{0.05cm}
        \interlinepenalty 10000 % Discourage LaTeX from adding page breaks in the env
        \begin{flushleft}
            \textbf{\large Requirement [#1] } \label{#1}
            \IfValueTF{#2} % If optionnal argument is passed
            {
                \hfill
                Validation strategy: #2
            }
            { } % No optional argument : print nothing
        \end{flushleft}
        \vspace{0.05cm}
        \begin{center}
        \begin{tabular}{|p{0.9\textwidth}|}
        \hline\\
    }
    {
        \\\\\hline
        \end{tabular}
        \end{center}
        \vspace{1cm}
        \interlinepenalty 0 % Reset the interline penalty for page breaks
    }

% Environment for authors / reviewers / etc. table on first page
\newenvironment{authors}
{} {}

% Chapter title format
\titleformat{\chapter}
{\normalfont\Large\bfseries}{\thechapter}{1em}{}
\titlespacing*{\chapter} {0pt}{3.5ex plus 1ex minus .2ex}{2.3ex plus .2ex}

% Fonts
\renewcommand{\rmdefault}{phv} % Arial
\renewcommand{\sfdefault}{phv} % Arial

% Header and footer

% \chapter command needs to be patched so it stops re-applying the 'plain' style
% See https://tex.stackexchange.com/questions/117328/fancyhdr-does-not-apply-same-header-footer-on-chapter-and-non-chapter-pages
\patchcmd{\chapter}{\thispagestyle{plain}}{\thispagestyle{fancy}}{}{}

% Setup the 'fancy' page style
\fancypagestyle{fancy}
{
    \fancyhf{} % remove everything
    \renewcommand{\headrulewidth}{2pt}
    \renewcommand{\footrulewidth}{1pt}

    \lhead  {
                \projectTitlePretty \\
                \documentTitlePretty
            }
    \chead  {
                \includegraphics [scale=0.6] {classification}
            }
    \rhead  {
                \documentUidPretty \\
                VERSION HERE \\
                \today
            }
    \rfoot  {
                \thepage~/~\pageref*{LastPage}
            }
    \cfoot  {
                \raisebox{-0.5\height}{\includegraphics [scale=0.6] {classification}}
            }
}

% Apply the style *everywhere* (except first page?)
\pagestyle{fancy}


% Glossary
\newglossarystyle{clong}
{
    \renewenvironment{theglossary}
        {\begin{longtable}{p{.3\linewidth}p{\glsdescwidth}}}%
        {\end{longtable}}
    \renewcommand*{\glossaryheader}{}%
    \renewcommand*{\glsgroupheading}[1]{}%
    \renewcommand*{\glossaryentryfield}[5]
        {
            \glstarget{##1}{##2} & ##3\glspostdescription\space ##5\\
        }
    \renewcommand*{\glossarysubentryfield}[6]
        {
            & \glstarget{##2}{\strut}##4\glspostdescription\space ##6\\
        }
}


% Generated constants
\input{snip/constants}

% Glossary generation
\input{snip/glossary} % TODO: from XML?


\newglossarystyle{clong}{%
 \renewenvironment{theglossary}%
     {\begin{longtable}{p{.3\linewidth}p{\glsdescwidth}}}%
     {\end{longtable}}%
  \renewcommand*{\glossaryheader}{}%
  \renewcommand*{\glsgroupheading}[1]{}%
  \renewcommand*{\glossaryentryfield}[5]{%
    \glstarget{##1}{##2} & ##3\glspostdescription\space ##5\\}%
  \renewcommand*{\glossarysubentryfield}[6]{%
     & \glstarget{##2}{\strut}##4\glspostdescription\space ##6\\}%
  %\renewcommand*{\glsgroupskip}{ & \\}%
}


% Document
\begin{document}
    % First page
    
\begin{titlepage}
    \begin{center}
    \hspace{0pt}
    \vfill

    {\LARGE \projectTitlePretty} \\
    \vspace{1.5cm}
    {\Huge \documentTitlePretty} \\
    \vspace{1.5cm}
    \documentUidPretty \hspace{1cm} \documentCategoryPretty

    \vfill
    \hspace{0pt}
    \end{center}
\end{titlepage}

% For some annoying reason, the titlepage env resets the page counter at start *and end*.
% Since we're not savages, we like having the page numbering actually match the pages...
\addtocounter{page}{1}


    \tableofcontents
    \pagebreak

    \listoffigures
    \pagebreak

    \listoftables
    \pagebreak

    \level{0}{Introduction}
    \level{1}{Purpose of the document}
    \blindtext[1]

    \level{1}{Document overview}

    This document is a \acrfull{srd}. The purpose of a \acrshort{srd} can be found in section \ref{glossary}.

    \level{1}{Acronyms and abreviations}\label{glossary}
%    \printglossary[style=long-booktabs]
    \printglossary[style=clong]
    \printglossary[type=\acronymtype]

    \level{0}{Documents}
    \level{1}{Applicable documents}
    \blindtext[1]

    \level{1}{Reference documents}
    \blindtext[1]

    \level{1}{Project reference documents}
    \blindtext[1]

    \level{0}{General description}
    \level{1}{Software item description}\label{software-item-description}

    The \gls{software} is described here.

    \level{1}{Function and purpose}
    \blindtext[1]

    \level{1}{General constraints}
    \blindtext[1]

    \level{1}{Model description}

    See figure~\ref{fig:architecture-diagram}.

    \begin{figure}[H]
    \includegraphics{architecture.png}
    \caption{Architecture diagram for SW.COMP1}
    \centering\label{fig:architecture-diagram}
    \end{figure}

    \blindtext[1]


    \level{0}{Specific requirements}

    \level{1}{Functionnal requirements}

    \level{2}{Input commands}

    \blindtext[1]

    \input{snip/SRD-REQ-00001}

    Note: see \ref{software-item-description} for details about this command.

    \level{2}{Performance requirements}
    \blindtext[1]

    \level{2}{Interface requirements}
    \blindtext[1]

    \input{snip/SRD-REQ-80000}
    Du texte en dessous pour préciser
    Une table...

    \level{2}{Operationnal requirements}
    \level{3}{Versionning}
    \input{snip/SRD-REQ-01000}

    \level{0}{Verification and validation}
    \level{1}{Requirements}
    \blindtext[1]

    \level{1}{Methods}
    \blindtext[1]

    \level{0}{Tracability matrices}\label{tracability-matrices}

    Test test
%    \input{snip/matrix_to_SP-PIDS}
%    \input{snip/SRD-REQ-00001.tex_table}

\end{document}
